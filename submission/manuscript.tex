\documentclass[11pt,]{article}
\usepackage{lmodern}
\usepackage{amssymb,amsmath}
\usepackage{ifxetex,ifluatex}
\usepackage{fixltx2e} % provides \textsubscript
\ifnum 0\ifxetex 1\fi\ifluatex 1\fi=0 % if pdftex
  \usepackage[T1]{fontenc}
  \usepackage[utf8]{inputenc}
\else % if luatex or xelatex
  \ifxetex
    \usepackage{mathspec}
  \else
    \usepackage{fontspec}
  \fi
  \defaultfontfeatures{Ligatures=TeX,Scale=MatchLowercase}
\fi
% use upquote if available, for straight quotes in verbatim environments
\IfFileExists{upquote.sty}{\usepackage{upquote}}{}
% use microtype if available
\IfFileExists{microtype.sty}{%
\usepackage{microtype}
\UseMicrotypeSet[protrusion]{basicmath} % disable protrusion for tt fonts
}{}
\usepackage[margin=1.0in]{geometry}
\usepackage{hyperref}
\hypersetup{unicode=true,
            pdftitle={Revisiting the Relationship between Short-Chain Fatty Acids, the Microbiota, and Colorectal Tumors},
            pdfborder={0 0 0},
            breaklinks=true}
\urlstyle{same}  % don't use monospace font for urls
\usepackage{graphicx,grffile}
\makeatletter
\def\maxwidth{\ifdim\Gin@nat@width>\linewidth\linewidth\else\Gin@nat@width\fi}
\def\maxheight{\ifdim\Gin@nat@height>\textheight\textheight\else\Gin@nat@height\fi}
\makeatother
% Scale images if necessary, so that they will not overflow the page
% margins by default, and it is still possible to overwrite the defaults
% using explicit options in \includegraphics[width, height, ...]{}
\setkeys{Gin}{width=\maxwidth,height=\maxheight,keepaspectratio}
\IfFileExists{parskip.sty}{%
\usepackage{parskip}
}{% else
\setlength{\parindent}{0pt}
\setlength{\parskip}{6pt plus 2pt minus 1pt}
}
\setlength{\emergencystretch}{3em}  % prevent overfull lines
\providecommand{\tightlist}{%
  \setlength{\itemsep}{0pt}\setlength{\parskip}{0pt}}
\setcounter{secnumdepth}{0}
% Redefines (sub)paragraphs to behave more like sections
\ifx\paragraph\undefined\else
\let\oldparagraph\paragraph
\renewcommand{\paragraph}[1]{\oldparagraph{#1}\mbox{}}
\fi
\ifx\subparagraph\undefined\else
\let\oldsubparagraph\subparagraph
\renewcommand{\subparagraph}[1]{\oldsubparagraph{#1}\mbox{}}
\fi

%%% Use protect on footnotes to avoid problems with footnotes in titles
\let\rmarkdownfootnote\footnote%
\def\footnote{\protect\rmarkdownfootnote}

%%% Change title format to be more compact
\usepackage{titling}

% Create subtitle command for use in maketitle
\newcommand{\subtitle}[1]{
  \posttitle{
    \begin{center}\large#1\end{center}
    }
}

\setlength{\droptitle}{-2em}
  \title{Revisiting the Relationship between Short-Chain Fatty Acids, the
Microbiota, and Colorectal Tumors}
  \pretitle{\vspace{\droptitle}\centering\huge}
  \posttitle{\par}
  \author{}
  \preauthor{}\postauthor{}
  \date{}
  \predate{}\postdate{}

\usepackage{helvet} % Helvetica font
\renewcommand*\familydefault{\sfdefault} % Use the sans serif version of the font
\usepackage[T1]{fontenc}

\usepackage[none]{hyphenat}

\usepackage{setspace}
\doublespacing
\setlength{\parskip}{1em}

\usepackage{lineno}

\usepackage{pdfpages}

\begin{document}
\maketitle

\vspace{35mm}

Running title: SCFAs and colorectal tumors

\vspace{35mm}

Marc A. Sze\({^1}\), Nicholas A. Lesniak\({^1}\), Mack T. Ruffin
IV\({^2}\), Patrick D. Schloss\({^1}\)\({^\dagger}\)

\vspace{40mm}

\(\dagger\) To whom correspondence should be addressed:
\href{mailto:pschloss@umich.edu}{\nolinkurl{pschloss@umich.edu}}

\(1\) Department of Microbiology and Immunology, University of Michigan,
Ann Arbor, MI 48109

\(2\) Department of Family Medicine and Community Medicine, Penn State
Hershey Medical Center, Hershey, PA

\newpage

\linenumbers

\subsection{Abstract}\label{abstract}

\textbf{Background.} Colorectal cancer (CRC) is increasing in prevalence
in individuals under 50 and because of this will be a continuing health
concern for the foreseeable future. The majority of the risk for
developing CRC is attributable to environmental factors. One of these
environmental factors is the microbiota, with certain bacterial
community members being associated with CRC and other taxa being
associated to individuals without tumors. Some of the bacterial species
in taxa associated to individuals without tumors can use fiber to
produce short-chain fatty acids (SCFAs) that inhibit tumor growth in
model systems. However, the data supporting the importance of SCFAs in
human CRC is less certain. Here, we test the hypothesis that SCFA
concentrations and taxa associated with their production are different
in individuals with colorectal tumors.

\textbf{Methods.} We analyzed a cross-sectional (n=490) and longitudinal
pre- and post-treatment (n=67) group for their fecal concentrations of
acetate, butyrate, and propionate. Analysis also included imputed gene
relative abundance with PICRUSt, metagenomic sequencing on a subset
(n=85) of the total cross-sectional group, and tumor classification and
SCFA prediction models using Random Forest.

\textbf{Results.} No difference in SCFA concentrations were found
between individuals without tumors and patients with adenomas or
carcinomas (P-value \textgreater{} 0.15). Using metagenomic sequencing,
there was also no difference in genes involved with SCFA synthesis
between individuals without tumors and patients with adenomas or
carcinomas (P-value \textgreater{} 0.70). Finally, there was no
difference between the ability of Random Forest models to predict
patients with adenomas or carcinomas versus individuals without tumors
(P-value \textgreater{} 0.05).

\textbf{Conclusions.} Although our data does not support the hypothesis
that fecal SCFA concentrations in patients in the general CRC population
are different, there still may be specific types of colorectal tumors
where SCFAs may be beneficial for treatment of CRC. Alternatively, our
observations also support the hypothesis that there may be other
metabolites or mechanisms (e.g.~bacterial niche exclusion) that may be
more protective against tumorigenesis and have not been thoroughly
investigated in the context of human CRC.

\newpage

\subsection{Introduction}\label{introduction}

Colorectal cancer (CRC) is currently the third leading cancer-related
cause of death within the US and the prevelance is increasing in
invidiauls under 50 years of age (1, 2). Although there is a genetic
component to the disease, the environment is considered a larger risk
factor for CRC (3). These environmental risk factors include but are not
limited to smoking cigarettes, diet, and the microbiota (4--6). Many of
these environmental risk factors, including the microbiota, are
modifiable. This has lead to the investigation of how the microbiota may
exacerbate or cause tumorigensis (7--9) and whether the bacterial
community is altered in CRC (10, 11). Many of these previous
case/control studies have identified resident bacterial taxa to be
decreased in patients with carcinoma tumors (11--13). Many of the
bacterial species from these resident taxa identified in these
case/control studies actively produce short-chain fatty acids (SCFAs)
from fiber that are part of our general diet (14). The most extensively
studied SCFAs are acetate, butyrate, and propionate (15). These SCFAs
are hypothesized to be the main metabolites involved with protection
against tumorigenesis and could help to reduce the risk of CRC.

Prior research suggests that SCFAs have promise in acting as an
anti-tumorigenic agent. Specific SCFAs have shown positive results
within model systems (16). For example, butyrate has been shown to
inhibit cancer cell growth in \emph{in vitro} systems (17).
Additionally, fiber supplementation in mouse models of CRC caused an
overall reduction in tumor burden while also increasing SCFA
concentrations (18). These exciting results in model systems suggest
that supplementation with food sources that bacteria use to create these
SCFAs may confer beneficial effects against CRC. However, it is
important to note that these model systems provide only preliminary
evidence towards the ability of SCFAs to reduce and treat tumors and the
studies reporting benefit in humans has been less convincing.

Overall, there is a lack of evidence on the benefit of increasing SCFA
concentrations to protect against CRC in human populations. The initial
case/control studies that investigated SCFA concentrations in CRC found
that patients with carcinomas had lower concentrations of acetate,
butyrate, and propionate versus patients with adenomas or individuals
without colon tumors (19). Although this would argue that increasing
SCFA concentrations could be protective against tumorigenesis, fiber
supplementation in randomized controlled trials have consistently failed
to protect against tumor recurrence (20). These findings argue against
the utility of treatments that aim to use SCFAs to reduce or protect
against tumorigenesis. Given the lack of clear evidence in human studies
of the benefit of SCFAs in CRC, there is a need for more investigation
into this area.

Our study fills some of the current gaps in the literature that relate
to the study of SCFAs and CRC in human populations. Specifically, by
using a seperate cohort it tests previous case/control findings on SCFA
concentrations in individuals with and without tumors. Additionally,
prior investigations grouped patients with adenoma and individuals
without tumors into a single group to compare against patients with
carcinomas. Despite doing this, the suggestion was made that a reduction
in SCFA concentration would also be observed between patients with
adenomas and indivdiuals without tumors. We also can test this
suggestion because we have a larger number of patients with adenomas
within our study and do not need to group patients with adenomas and
individuals without tumors into a single group. Additionally, we build
upon these observations by assessing the utility of using SCFAs and
Operational Taxonomic Units (OTUs) as a risk stratification tool of
colorectal tumors (adenoma or carcinoma). We also investigate whether
OTUs that are most important to these models are closely associated with
the classification of SCFA concentrations. Collectively, this study
provides important information on the replicability of previous findings
in humans by extensively studying how SCFAs are associated with
colorectal tumors.

To accomplish this task we directly measured the concentration of
acetate, butyrate, and propionate within fecal samples for two different
groups. The first group had a sample obtained at a single cross
sectional point in time while the second group had samples obtained
before (pre-) and after (post-)treatment for colorectal tumors. To
provide further support for our SCFA findings we also used PICRUSt (21)
and metagenomic sequencing to investigate differences in genes involved
with SCFA synthesis between individuals without tumors, patients with
adenomas, and patients with carcinomas. Next, we investigated whether
taxa associated with SCFA production were important to disease
classification models. First, using the cross-sectional data, we
analyzed the number of correlations between OTU relative abundance and
SCFA concentrations across individuals without tumors and patients with
adenomas or carcinomas. Second, we assessed the affect adding SCFA
concentrations to OTU data had on classification of patients with
adenomas or carcinomas using the Random Forest algorithm (22). Third, we
analyzed how well 16S rRNA gene sequencing predicts SCFA concentrations.
Finally, we compared whether SCFA concentrations replace important taxa
in disease models and whether the same OTUs are the most important
variables used to classify disease and SCFA concentration. Collectively,
this investigation provides additional information as to whether SCFAs
are decreased in patients with colorectal tumors and provides context as
to whether targeting taxa to increase SCFA concentrations is a viable
option to protect against colon tumorigenesis.

\newpage

\subsection{Results}\label{results}

\textbf{Decreased SCFA concentrations are not associated with adenoma or
carcinoma tumors.} We used high-performance liquid chromatography (HPLC)
to measure acetate, butyrate, and propionate concentrations of frozen
fecal samples from 490 individuals at a cross-sectional point in time.
There was no difference between individuals without colon tumors (n=172)
and patients with either an adenoma (n=198) or carcinoma (n=120) for any
of the SCFAs measured after multiple comparison correction (P-value
\textgreater{} 0.15) {[}Figure 1A - 1C{]}. We next measured the
concentration of SCFAs in 67 patients with an adenoma (n=41) or
carcinoma (n=26) in which we had pre- and post-treatment fecal samples.
Although there was a general trend for increasing acetate, butyrate, and
propionate concentrations following treatment for tumors, there was no
significant difference between pre- and post-treatment for either
patients with adenomas (P-value \textgreater{} 0.20) or carcinomas
(P-value \textgreater{} 0.80) {[}Figure 1D - 1F{]}.

\textbf{Gene abundance for enzymes involved in SCFA synthesis are the
same for individuals without tumors and patients with adenomas or
carcinomas.} In order to provide further confirmation and support of our
SCFA concentration results we investigated the genes encoding specific
enzymes involved with SCFA synthesis {[}Table S1{]}. Using this list of
specific genes {[}Table S1{]}, we looked for differences in gene
abundance between individuals without colon tumors and patients with
adenomas or carcinomas. Although we intended to analyze all the genes in
the list {[}Table S1{]}, not all of the KEGG genes were identified
during our analysis. We first analyzed imputed gene relative abundance,
calculated from our OTU data that was generated from 16S rRNA gene
sequencing. We found no difference in any of the imputed gene relative
abundances for enzymes involved with acetate, butyrate, or propionate
synthesis (P-value \textgreater{} 0.90) {[}Table S2{]}. Since butyrate
is one of the most studied SCFAs in the context of CRC, we visualized
the observed lack of difference between indviduals without tumors and
patients with adenomas or carcinomas in gene abundance for enzymes
involved with SCFA synthesis using the butyrate kinase gene {[}Figure
2A{]}. Additionally, using a paired Wilcoxon rank-sum test, there also
was no difference in imputed gene relative abundance between pre- and
post-treatment samples for any genes involved with SCFA synthesis
(P-value \textgreater{} 0.70) {[}Table S3{]}. Next, we took a subset of
these 490 fecal samples (n=85) and used metagenomic sequencing to
confirm these results. Like the imputed gene results, metagenomic
analysis found that there was no difference in any of the genes involved
in SCFA synthesis between individuals without colon tumors (n=29) and
patients with adenomas (n=28) or carcinomas (n=28) (P-value
\textgreater{} 0.70) {[}Table S4{]}. This similarity between indviduals
without tumors and patients with adenomas or carcinomas is highlighted
again by visualizing the gene abundance for the butyrate kinase gene
{[}Figure 2B{]}. These observations provide evidence that gene
prevalence for enzymes involved in SCFA synthesis does not change due to
colorectal tumors and provides further support for our original SCFA
concentration observations.

\textbf{The number of OTUs positively associated with SCFA concentration
were similar between individuals without tumors and patients with
adenomas or carcinomas.} Having found no difference between individuals
without tumors and patients with adenomas or carcinomas in SCFA
concentrations or genes encoding enzymes involved with SCFA synthesis,
we next investigated if specific OTUs correlated with SCFA
concentrations. The main goal of this analysis was to identify if there
were OTUs that were significantly associated with SCFA concentrations
and if this was different between individuals that did not have tumors,
had an adenoma, or had a carcinoma. To accomplish this we used Spearmans
rho, a non-parametric measure of association, and tested if there was a
correlation that was signficantly greater than zero. We found that taxa
from \emph{Clostridiales}, \emph{Lachnospiraceae}, and
\emph{Ruminococcaceae} dominated statistically significant OTU
correlations {[}Figure 3 \& Table S5{]}. There was a noticeably higher
number of significant negative correlations associated with patients
with adenomas for all SCFAs tested {[}Figure 3{]}. In particular, OTUs
from the \emph{Ruminococcaceae} family had the largest share of these
negative correlations within patients with adenomas {[}Figure 3{]}.
Patients with adenomas also had more positive correlations between OTUs
and SCFA concentrations, but their total number was more similar to
individuals without tumors or patients with carcinomas versus the
analogous comparison for the number of negative correlations {[}Figure
3{]}. Additionally, the number of positive correlations between OTUs and
SCFA concentrations was similar between individuals without tumors and
patients with a carcinoma {[}Figure 3{]}. Finally, when we used high/low
SCFA groups based on the overall median concentration for each SCFA
instead of SCFA concentrations a similar pattern was still observed
{[}Figure S1 \& Table S6{]}. Overall, these results suggest that the
resident taxa that may change the most due to colon tumors may not be
ones that are responsible for the production of acetate, butyrate, or
propionate.

\textbf{SCFA concentrations do not replace important
\emph{Clostridiales}, \emph{Lachnospiraceae}, and \emph{Ruminococcaceae}
OTUs in Random Forest models built to classify tumors.} Despite the lack
of difference in positive correlations between OTUs and SCFA
concentrations between individuals with and without tumors, OTUs
associated with SCFA concentrations could still be the most important
variables to Random Forest models built to classify patients with
adenomas or carcinomas. We tested this by using the Random Forest
algorithm to build models with OTU abundance data or OTU abundances and
SCFA concentrations to classify normal versus adenoma and normal versus
carcinoma fecal samples. With these models we compared whether OTUs with
taxonomic classification to \emph{Clostridiales},
\emph{Lachnospiraceae}, and \emph{Ruminococcaceae} remained when SCFA
concentrations were added to the model. Additionally, we also compared
whether any of the important OTUs that remained also had significant
correlations with SCFA concentrations. Both our adenoma and carcinoma
models classified patients with a similar degree of success, as measured
by the area under the curve (AUC) (P-value \textgreater{} 0.05)
{[}Figure 4A \& 4D{]}. After the addition of SCFA concentrations to the
adenoma or carcinoma models, many OTUs with taxonomic classification to
\emph{Clostridiales}, \emph{Lachnospiraceae}, and \emph{Ruminococcaceae}
remained as important variables to the model {[}Figure Figure 4B-C \&
4E-F{]}. After adding SCFA concentrations to the adenoma model, there
were only 2 OTUs significantly associated with SCFA concentration that
were part of the top 10 most important variables and both were postively
associated with acetate, butyrate, or propionate concentrations. When
SCFA concentrations were added to the carcinoma model, only 1 OTU that
was associated with SCFA concentration remained as part of the top 10
most important variables and it was negatively associated with acetate
and butyrate concentrations. In combination with the previous results on
OTU correlations, these observations provide additional evidence that
the resident taxa that are associated with protection against
tumorigeneis are not ones associated with acetate, butyrate, or
propionate production.

\textbf{The most important OTUs in Random Forest models built to
classify SCFA concentrations or tumors are not the same.} It is possible
that due to the way the Random Forest algorithm works, OTUs associated
with SCFA concentrations could be downweighted in importance within the
adenoma or carcinoma models when SCFA concentrations are included. To
test if this is the case we used OTU data and built Random Forest models
to classify SCFA concentrations. Overall, the correlation between the
predicted and actual SCFA concentrations were moderately associated with
each other {[}Figure 5A{]}. Additionally, because the training set
R\textsuperscript{2} was always higher than the test set
R\textsuperscript{2}, all SCFA concentration models tended to be over
fit, suggesting that rarer taxa were important for these classifications
{[}Figure 5A{]}. There also was a difference in accuracy based on
whether the fecal sample was from an individual without tumors or from
patients with adenomas or carcinomas {[}Figure 5B{]}. When comparing the
adenoma or carcinoma model to the SCFA concentration models there was
minimal overlap between these model's most important OTUs {[}Figure
4B-C, 4E-F and 5C-E{]}. The only OTU that was in the top 10 most
important variables and had overlap between the models was OTU00167
(\emph{Clostridiales}) {[}Figure 4B-C, 4E-F, 5C-E{]}. Similar
observations were made when using high/low SCFA groups based on the
median SCFA concentration {[}Figure S2{]}. Collectively, these
observations provide evidence that it is possible to identify specific
OTUs associated with higher SCFA concentrations and accordingly these
OTUs belong to taxa known to produce acetate, butyrate, and propionate.
Although it is possible to identify OTUs associated with SCFA
production, overall, our results do not support the hypothesis that SCFA
concentration or OTUs associated with their production are different
between individuals with no tumors and patients with adenomas or
carcinomas.

\newpage

\subsection{Discussion}\label{discussion}

The observations from this study do not support the hypothesis that SCFA
concentrations are different in individuals with tumors. Whether we
directly measured the SCFA concentration or investigated genes that
encoded enzymes used for their production, no difference could be
identified between individuals without tumors and patients with adenomas
or carcinomas {[}Figure 1 \& 2{]}. Although there were differences in
the number of significant correlations between SCFA concentration and
OTU relative abundance based on whether individuals did not have tumors,
had an adenoma, or had a carcinoma, SCFA concentrations did not provide
increased model accuracy for tumor classification {[}Figure 3-4 \&
S1{]}. In models with SCFA concentrations included, many OTUs that
classified to \emph{Clostridiales}, \emph{Lachnospiraceae}, and
\emph{Ruminococcaceae} remained as important variables for the model
{[}Figure 4{]}. Additionally, when models using OTU relative abundance
to classify SCFA concentrations were assessed, the OTUs that classified
to \emph{Clostridiales}, \emph{Lachnospiraceae}, and
\emph{Ruminococcaceae} were not the same as the OTUs that classified to
these taxa in the tumor models {[}Figure 4-5{]}. Collectively, our
observations suggest that the resident taxa from \emph{Clostridiales},
\emph{Lachnospiraceae}, and \emph{Ruminococcaceae} that are different
between individuals without tumors and patients with adenomas or
carcinomas, are not the same as those involved with SCFA production.

Although SCFAs have been shown to be anti-tumorigenic, most of these
studies have been performed in model systems (16, 17). Many of the
\emph{in vivo} studies use proxies such as fiber supplementation rather
than SCFAs directly (14). Although it is well known that breakdown
products from gut bacteria results in SCFA production, fiber effects on
tumorigenesis may be through other mechanisms in these \emph{in vivo}
model systems. Additionally, the observations in humans on the benefit
of SCFAs in preventing tumorigenesis have been mixed. In previous
case/control studies lower SCFA concentrations were observed in patients
with carcinomas versus those without carcinomas (19). Yet, this is in
contrast to multiple randomized-controlled trials that have found no
difference in tumor recurrence between patients who do and do not get
fiber supplementation (20, 23). In contrast to the \emph{in vivo} model
findings, the observations made in these randomized-controlled trials
would suggest that SCFAs do not prevent or slow tumorigenesis. One
reason for these results is that SCFA concentrations and responses to
fiber vary quite a bit between healthy individuals (24). This
information taken together with our observations would suggest that
either individuals who do not respond to fiber supplementation would
need to acquire these bacteria to achieve a benefit or that SCFAs
provide little to no benefit as an anti-tumorigenic compound in
colorectal cancer.

Another possible alternative explanation as to why no difference in SCFA
concentration between individuals without tumors and patients with
adenomas or carcinomas was observed, could be because only certain types
of colorectal cancers are affected by SCFAs. One limitation of current
research into the effect of SCFAs and the microbiota in CRC has been
that all tumors are treated as the same type. However, there are known
differences in the types of mutations that occur (25) and treating all
tumors as equal may actually hide any benefit that could be found in
certain subsets of individuals. Similar to the idea of using
immunotherapy as a targeted treatment option for specific tumors (26),
SCFAs may have beneficial effects for distinct types of colorectal
tumors. Future research will need to test if this is a valid hypothesis.
Regardless of this limitation, our results in combination to previous
randomized controlled trials on fiber supplementation (20, 23) suggests
that using SCFAs as a general treatment for colorectal cancer is
unlikely to provide a reduction in tumorigenesis.

One possible technical limitation is that a fecal sample may not be an
ideal type of bio-specimen and that the effect SCFAs have on
tumorigenesis is only detected in the colon. However, this is unlikely
to be a major confounder. First, most \emph{in vivo} studies, as well as
human studies, have used fecal material in their analysis (18, 19).
Second, previous studies that measure SCFA changes after fiber
supplementation use fecal material to track these responses with a great
deal of success (24). Although there are limitations with the current
research on SCFAs and colorectal tumors, technical limitations are less
likely to be the cause of this. Clearly, our observations provide robust
evidence that these specific metabolites may not be protective or used
as a general treatment option in colorectal cancer. Yet, taxa that are
associated with SCFA production are consistently higher in individuals
without colon tumors than patients with carcinomas (10, 11, 27).

The potential protection against colorectal cancers may not be from
SCFAs, even though taxa associated with their production are higher in
individuals without tumors versus patients with carcinomas (10, 11, 27).
Our data is consistent with the literature despite finding no difference
in SCFA concentrations between individuals without tumors and patients
with adenomas or carcinomas. In particular, we observed that the
important OTUs in our tumor models were to taxa that are associated with
SCFA production but that these OTUs were not themselves associated with
acetate, butyrate, or propionate production. One of the clearest results
to highlight this within our study were the results showing that
different OTUs from the same taxonomic classification are in tumor and
SCFA Random Forest models {[}Figure 4 \& 5{]}. This leads to the
possibility that protection may be through other routes. First, there
could be a different pathway or other less extensively studied
metabolites that provide the necessary protection against tumorigenesis.
Alternatively, previous research has shown that non-resident oral
microbes, such as \emph{Fusobacterium}, \emph{Porphyromonas},
\emph{Parvimonas}, and \emph{Peptostreptococcus} are associated with
carcinomas (6, 12, 13) and can increase inflammation (28). When these
non-resident oral microbes increase inflammation they change the
existing environment to one in which is more hospitable for themselves
(28). It is possible that these resident microbes provide protection not
through a metabolite but instead through active exclusion of these
non-resident oral microbes which in turn prevents them from establishing
themselves within the community. Overall, even though we did not find
lower concentrations of SCFAs associated with colorectal tumors, we
think that our results open up many new and exciting avenues for
investigation into how metabolites and the microbiota can prevent
tumorigenesis.

\newpage

\subsection{Conclusions}\label{conclusions}

Our observations found no difference in SCFA concentration or in genes
encoding enzymes involved in their production between individuals
without colon tumors and patients with either adenoma or carcinoma
tumors. The SCFA concentration also did not improve classification of
adenoma or carcinoma tumors. The most important OTUs to these models
classifying patients with adenomas or carcinomas were mostly not
associated with SCFA production. Although these results are different
than other reports in the literature, they do align with the randomized
controlled trials that have tested fiber use in preventing colorectal
tumor recurrence. Additionally, these observations suggest that resident
microbes that are not involved in SCFA production may be the important
resident community members involved with preventing tumorigenesis. By
focusing on alternative mechanisms that are associated with these
non-SCFA producing resident microbes or SCFA effects on distinct
colorectal cancer groups the identification of more promising
therapeutic options for use in treating colorectal cancer may be found.

\newpage

\subsection{Materials and Methods}\label{materials-and-methods}

\textbf{Study design and sampling.} The overall protocol has been
described in detail previously (29, 30). In brief, this study used fecal
samples obtained at either a single cross-sectional time point (n=490)
or from before (pre-) and after (post-) treatment of a patient's tumor
(adenoma n =41 and carcinoma n = 26). For patients undergoing treatment
for their tumor the length of time between their initial and follow up
sample ranged from 188 - 546 days. Our use of treatment has been
previously defined as encompassing removal of a tumor with or without
chemotherapy and radiation (29). Diagnosis of tumor was made by
colonoscopic examination and histopathological review of biopsies
obtained (29, 30). The University of Michigan Institutional Review Board
approved the study and informed consent was obtained from all
participants in accordance to the guidelines set out by the Helsinki
Declaration.

\textbf{Measuring specific SCFAs.} Our protocol for the measurement of
acetate, butyrate, and propionate followed a previously published
protocol that used a High-Performance Liquid Chromatography (HPLC)
machine (24). The following changes to this protocol included the use of
frozen fecal samples suspended in 1ml of PBS instead of fecal
suspensions in DNA Genotek OmniGut tubes, and the use of the actual
weight of fecal samples instead of the average weight for SCFA
concentration normalizations. These methodological changes did not
affect the overall median concentrations of these SCFAs between the two
studies (see Table 1 (24) and Figure 1 here).

\textbf{16s rRNA gene sequencing.} The workflow and processing have been
previously described (29, 31, 32). In brief, sequences were quality
filtered and contigs created from the paired end reads. Any sequences
with ambiguous base calls were discarded. Contigs were then checked for
matches to the V4 region of the 16S rRNA gene using the SILVA database
(33). Chimeras were identified and removed using UCHIME and OTUs
clustered at 97\% similarity (34). The major differences from these
previous reports include: the use of version 1.39.5 of the mothur
software package and clustering Operational Taxonomic Units (OTUs) at
97\% similarity using the OptClust algorithm (35).

\textbf{Generating imputed metagenomes.} The use of PICRUSt version
1.1.2 with the recommended standard operating protocol (21) was used.
Briefly, the mothur shared file and metadata was converted into a biom
formatted table using the biom convert function, the subsequent biom
file was processed with the `normalize\_by\_copy\_number.py' function,
and subsequent imputed metagenomes created using the
`predict\_metagenomes.py' function.

\textbf{Obtaining Operational Protein Families from metagenomes.} A
subset of the cross-sectional group (n=490) containing a total of 85
individuals (normal n=29, adenoma n=28, and carcinoma n=28) was shotgun
sequenced on an Illumina HiSeq using 125 bp paired end reads and a
previously described method (36). Briefly, the sequences were quality
filtered and sequences aligning to the human genome were removed prior
to contig assembly with MEGAHIT (37). Open Reading Frames (ORFs) were
identified using Prodigal (38), counts generated using Diamond (39),
subsequent clustering into Operational Protein Families (OPFs) used
mmseq2 (40), and OPF alignment used the KEGG database (41).

\textbf{Pulling genes involved with SCFA synthesis.} Specific genes
located near the end of the pathways involved in the synthesis of
acetate, butyrate, and propionate were analyzed for any differences
between individuals with normal colons and those with tumors. These
genes were based on pathways from KEGG as well as previous research (41,
42) and a list can be found in the supplemental material {[}Table S1{]}.

\textbf{Random Forest models.} The model was first trained on 80\% of
the data and then tested on the held out 20\% (80/20 split) using the
Random Forest algorithm for classification and regression models via the
caret package (22, 43). This was repeated on 100 different 80/20 splits
of the data to generate a reasonable range for the AUC of the model. The
reported AUCs, unless otherwise specified, are for the test sets. The
classification models were built to group normal versus adenoma, normal
versus carcinoma, and high versus low SCFA concentrations. The
regression models were built to classify the SCFA concentrations of
acetate, butyrate, and propionate regardless of disease status.

\textbf{Statistical analysis workflow.} All analysis was performed using
the statistical language R (44). Generally, a Kruskal-Walis rank sum
test with a Dunn's post-hoc test was used to assess differences between
the groups used. Where appropriate Benjamini-Hochberg was used to
correct for multiple comparisons (45). First, we assessed differences in
SCFA concentrations measured by HPLC between individuals with normal
colons and patients with tumors (adenoma or carcinoma). We then analyzed
whether SCFA concentrations changed in patients with an adenoma or
carcinoma pre- versus post-treatment. Next, the imputed gene counts of
important mediators of SCFA synthesis was tested. Additionally, the
counts generated for OPFs that matched important genes involved with
SCFA creation were analyzed. From here we analyzed the number of
significant positive and negative correlations between OTU relative
abundance and SCFA concentrations in individuals without tumors and
patients with adenomas or carcinomas using Spearman's rho. Next, we
assessed whether OTUs alone or OTUs and SCFAs were better able to
classify individuals with and without tumors using Random Forest models.
Finally, models to classify high or low SCFA concentration based on the
median of each SCFA or the actual concentration using 16S rRNA gene
sequencing data was created using the Random Forest algorithm. For all
Random Forest models, the assessment of the most important variables was
based on the top 10 features (OTUs or SCFAs) using the mean decrease in
accuracy.

\newpage

\subsection{Acknowledgements}\label{acknowledgements}

The authors thank the Great Lakes-New England Early Detection Research
Network for providing the fecal samples that were used in this study. We
would also like to thank Kwi Kim and Thomas M Schmidt for their help in
running the short-chain fatty acid analysis on the High-Performance
Liquid Chromatography machine at the University of Michigan. Salary
support for Marc A. Sze came from the Canadian Institute of Health
Research and NIH grant UL1TR002240. Salary support for Patrick D.
Schloss came from NIH grants P30DK034933 and 1R01CA215574.

\newpage

\subsection{References}\label{references}

\hypertarget{refs}{}
\hypertarget{ref-Haggar2009}{}
1. \textbf{Haggar F}, \textbf{Boushey R}. 2009. Colorectal cancer
epidemiology: Incidence, mortality, survival, and risk factors. Clinics
in Colon and Rectal Surgery \textbf{22}:191--197.
doi:\href{https://doi.org/10.1055/s-0029-1242458}{10.1055/s-0029-1242458}.

\hypertarget{ref-Siegel2016}{}
2. \textbf{Siegel RL}, \textbf{Miller KD}, \textbf{Jemal A}. 2016.
Cancer statistics, 2016. CA: A Cancer Journal for Clinicians
\textbf{66}:7--30.
doi:\href{https://doi.org/10.3322/caac.21332}{10.3322/caac.21332}.

\hypertarget{ref-Lichtenstein2000}{}
3. \textbf{Lichtenstein P}, \textbf{Holm NV}, \textbf{Verkasalo PK},
\textbf{Iliadou A}, \textbf{Kaprio J}, \textbf{Koskenvuo M},
\textbf{Pukkala E}, \textbf{Skytthe A}, \textbf{Hemminki K}. 2000.
Environmental and heritable factors in the causation of cancer analyses
of cohorts of twins from sweden, denmark, and finland. New England
Journal of Medicine \textbf{343}:78--85.
doi:\href{https://doi.org/10.1056/nejm200007133430201}{10.1056/nejm200007133430201}.

\hypertarget{ref-FlissIsakov2017}{}
4. \textbf{Fliss-Isakov N}, \textbf{Zelber-Sagi S}, \textbf{Webb M},
\textbf{Halpern Z}, \textbf{Kariv R}. 2017. Smoking habits are strongly
associated with colorectal polyps in a population-based case-control
study. Journal of Clinical Gastroenterology 1.
doi:\href{https://doi.org/10.1097/mcg.0000000000000935}{10.1097/mcg.0000000000000935}.

\hypertarget{ref-Lee2015}{}
5. \textbf{Lee J}, \textbf{Jeon JY}, \textbf{Meyerhardt JA}. 2015. Diet
and lifestyle in survivors of colorectal cancer. Hematology/Oncology
Clinics of North America \textbf{29}:1--27.
doi:\href{https://doi.org/10.1016/j.hoc.2014.09.005}{10.1016/j.hoc.2014.09.005}.

\hypertarget{ref-Kostic2011}{}
6. \textbf{Kostic AD}, \textbf{Gevers D}, \textbf{Pedamallu CS},
\textbf{Michaud M}, \textbf{Duke F}, \textbf{Earl AM}, \textbf{Ojesina
AI}, \textbf{Jung J}, \textbf{Bass AJ}, \textbf{Tabernero J},
\textbf{Baselga J}, \textbf{Liu C}, \textbf{Shivdasani RA},
\textbf{Ogino S}, \textbf{Birren BW}, \textbf{Huttenhower C},
\textbf{Garrett WS}, \textbf{Meyerson M}. 2011. Genomic analysis
identifies association of fusobacterium with colorectal carcinoma.
Genome Research \textbf{22}:292--298.
doi:\href{https://doi.org/10.1101/gr.126573.111}{10.1101/gr.126573.111}.

\hypertarget{ref-Zackular2013}{}
7. \textbf{Zackular JP}, \textbf{Baxter NT}, \textbf{Iverson KD},
\textbf{Sadler WD}, \textbf{Petrosino JF}, \textbf{Chen GY},
\textbf{Schloss PD}. 2013. The gut microbiome modulates colon
tumorigenesis. mBio \textbf{4}:e00692--13--e00692--13.
doi:\href{https://doi.org/10.1128/mbio.00692-13}{10.1128/mbio.00692-13}.

\hypertarget{ref-Baxter2014}{}
8. \textbf{Baxter NT}, \textbf{Zackular JP}, \textbf{Chen GY},
\textbf{Schloss PD}. 2014. Structure of the gut microbiome following
colonization with human feces determines colonic tumor burden.
Microbiome \textbf{2}:20.
doi:\href{https://doi.org/10.1186/2049-2618-2-20}{10.1186/2049-2618-2-20}.

\hypertarget{ref-Zackular2015}{}
9. \textbf{Zackular JP}, \textbf{Baxter NT}, \textbf{Chen GY},
\textbf{Schloss PD}. 2015. Manipulation of the gut microbiota reveals
role in colon tumorigenesis. mSphere \textbf{1}:e00001--15.
doi:\href{https://doi.org/10.1128/msphere.00001-15}{10.1128/msphere.00001-15}.

\hypertarget{ref-Shah2017}{}
10. \textbf{Shah MS}, \textbf{DeSantis TZ}, \textbf{Weinmaier T},
\textbf{McMurdie PJ}, \textbf{Cope JL}, \textbf{Altrichter A},
\textbf{Yamal J-M}, \textbf{Hollister EB}. 2017. Leveraging
sequence-based faecal microbial community survey data to identify a
composite biomarker for colorectal cancer. Gut \textbf{67}:882--891.
doi:\href{https://doi.org/10.1136/gutjnl-2016-313189}{10.1136/gutjnl-2016-313189}.

\hypertarget{ref-meta_analysis_crc_Sze2018}{}
11. \textbf{Sze MA}, \textbf{Schloss PD}. 2018. Leveraging existing 16S
rRNA gene surveys to identify reproducible biomarkers in individuals
with colorectal tumors.
doi:\href{https://doi.org/10.1101/285486}{10.1101/285486}.

\hypertarget{ref-Zeller2014}{}
12. \textbf{Zeller G}, \textbf{Tap J}, \textbf{Voigt AY},
\textbf{Sunagawa S}, \textbf{Kultima JR}, \textbf{Costea PI},
\textbf{Amiot A}, \textbf{Bohm J}, \textbf{Brunetti F},
\textbf{Habermann N}, \textbf{Hercog R}, \textbf{Koch M},
\textbf{Luciani A}, \textbf{Mende DR}, \textbf{Schneider MA},
\textbf{Schrotz-King P}, \textbf{Tournigand C}, \textbf{Nhieu JTV},
\textbf{Yamada T}, \textbf{Zimmermann J}, \textbf{Benes V},
\textbf{Kloor M}, \textbf{Ulrich CM}, \textbf{Knebel Doeberitz M von},
\textbf{Sobhani I}, \textbf{Bork P}. 2014. Potential of fecal microbiota
for early-stage detection of colorectal cancer. Molecular Systems
Biology \textbf{10}:766--766.
doi:\href{https://doi.org/10.15252/msb.20145645}{10.15252/msb.20145645}.

\hypertarget{ref-Baxter2016}{}
13. \textbf{Baxter NT}, \textbf{Ruffin MT}, \textbf{Rogers MAM},
\textbf{Schloss PD}. 2016. Microbiota-based model improves the
sensitivity of fecal immunochemical test for detecting colonic lesions.
Genome Medicine \textbf{8}.
doi:\href{https://doi.org/10.1186/s13073-016-0290-3}{10.1186/s13073-016-0290-3}.

\hypertarget{ref-Holscher2017}{}
14. \textbf{Holscher HD}. 2017. Dietary fiber and prebiotics and the
gastrointestinal microbiota. Gut Microbes \textbf{8}:172--184.
doi:\href{https://doi.org/10.1080/19490976.2017.1290756}{10.1080/19490976.2017.1290756}.

\hypertarget{ref-Louis2016}{}
15. \textbf{Louis P}, \textbf{Flint HJ}. 2016. Formation of propionate
and butyrate by the human colonic microbiota. Environmental Microbiology
\textbf{19}:29--41.
doi:\href{https://doi.org/10.1111/1462-2920.13589}{10.1111/1462-2920.13589}.

\hypertarget{ref-test_OKeefe2016}{}
16. \textbf{O'Keefe SJD}. 2016. Diet, microorganisms and their
metabolites and colon cancer. Nature Reviews Gastroenterology \&
Hepatology \textbf{13}:691--706.
doi:\href{https://doi.org/10.1038/nrgastro.2016.165}{10.1038/nrgastro.2016.165}.

\hypertarget{ref-Encarnao2018}{}
17. \textbf{Encarnação JC}, \textbf{Pires AS}, \textbf{Amaral RA},
\textbf{Gonçalves TJ}, \textbf{Laranjo M}, \textbf{Casalta-Lopes JE},
\textbf{Gonçalves AC}, \textbf{Sarmento-Ribeiro AB}, \textbf{Abrantes
AM}, \textbf{Botelho MF}. 2018. Butyrate, a dietary fiber derivative
that improves irinotecan effect in colon cancer cells. The Journal of
Nutritional Biochemistry \textbf{56}:183--192.
doi:\href{https://doi.org/10.1016/j.jnutbio.2018.02.018}{10.1016/j.jnutbio.2018.02.018}.

\hypertarget{ref-Bishehsari2018}{}
18. \textbf{Bishehsari F}, \textbf{Engen P}, \textbf{Preite N},
\textbf{Tuncil Y}, \textbf{Naqib A}, \textbf{Shaikh M}, \textbf{Rossi
M}, \textbf{Wilber S}, \textbf{Green S}, \textbf{Hamaker B},
\textbf{Khazaie K}, \textbf{Voigt R}, \textbf{Forsyth C},
\textbf{Keshavarzian A}. 2018. Dietary fiber treatment corrects the
composition of gut microbiota, promotes SCFA production, and suppresses
colon carcinogenesis. Genes \textbf{9}:102.
doi:\href{https://doi.org/10.3390/genes9020102}{10.3390/genes9020102}.

\hypertarget{ref-Ohigashi2013}{}
19. \textbf{Ohigashi S}, \textbf{Sudo K}, \textbf{Kobayashi D},
\textbf{Takahashi O}, \textbf{Takahashi T}, \textbf{Asahara T},
\textbf{Nomoto K}, \textbf{Onodera H}. 2013. Changes of the intestinal
microbiota, short chain fatty acids, and fecal pH in patients with
colorectal cancer. Digestive Diseases and Sciences
\textbf{58}:1717--1726.
doi:\href{https://doi.org/10.1007/s10620-012-2526-4}{10.1007/s10620-012-2526-4}.

\hypertarget{ref-Yao2017}{}
20. \textbf{Yao Y}, \textbf{Suo T}, \textbf{Andersson R}, \textbf{Cao
Y}, \textbf{Wang C}, \textbf{Lu J}, \textbf{Chui E}. 2017. Dietary fibre
for the prevention of recurrent colorectal adenomas and carcinomas.
Cochrane Database of Systematic Reviews.
doi:\href{https://doi.org/10.1002/14651858.cd003430.pub2}{10.1002/14651858.cd003430.pub2}.

\hypertarget{ref-Langille2013}{}
21. \textbf{Langille MGI}, \textbf{Zaneveld J}, \textbf{Caporaso JG},
\textbf{McDonald D}, \textbf{Knights D}, \textbf{Reyes JA},
\textbf{Clemente JC}, \textbf{Burkepile DE}, \textbf{Thurber RLV},
\textbf{Knight R}, \textbf{Beiko RG}, \textbf{Huttenhower C}. 2013.
Predictive functional profiling of microbial communities using 16S rRNA
marker gene sequences. Nature Biotechnology \textbf{31}:814--821.
doi:\href{https://doi.org/10.1038/nbt.2676}{10.1038/nbt.2676}.

\hypertarget{ref-randomforest_citation_2002}{}
22. \textbf{Liaw A}, \textbf{Wiener M}. 2002. Classification and
regression by randomForest. R News \textbf{2}:18--22.

\hypertarget{ref-Schatzkin2000}{}
23. \textbf{Schatzkin A}, \textbf{Lanza E}, \textbf{Corle D},
\textbf{Lance P}, \textbf{Iber F}, \textbf{Caan B}, \textbf{Shike M},
\textbf{Weissfeld J}, \textbf{Burt R}, \textbf{Cooper MR},
\textbf{Kikendall JW}, \textbf{Cahill J}, \textbf{Freedman L},
\textbf{Marshall J}, \textbf{Schoen RE}, \textbf{Slattery M}. 2000. Lack
of effect of a low-fat, high-fiber diet on the recurrence of colorectal
adenomas. New England Journal of Medicine \textbf{342}:1149--1155.
doi:\href{https://doi.org/10.1056/nejm200004203421601}{10.1056/nejm200004203421601}.

\hypertarget{ref-scfa_measures_venkataraman2016}{}
24. \textbf{Venkataraman A}, \textbf{Sieber JR}, \textbf{Schmidt AW},
\textbf{Waldron C}, \textbf{Theis KR}, \textbf{Schmidt TM}. 2016.
Variable responses of human microbiomes to dietary supplementation with
resistant starch. Microbiome \textbf{4}.
doi:\href{https://doi.org/10.1186/s40168-016-0178-x}{10.1186/s40168-016-0178-x}.

\hypertarget{ref-Fearon1990}{}
25. \textbf{Fearon ER}, \textbf{Vogelstein B}. 1990. A genetic model for
colorectal tumorigenesis. Cell \textbf{61}:759--767.
doi:\href{https://doi.org/10.1016/0092-8674(90)90186-i}{10.1016/0092-8674(90)90186-i}.

\hypertarget{ref-Thomas2016}{}
26. \textbf{Thomas X}, \textbf{Heiblig M}. 2016. The development of
agents targeting the BCR-ABL tyrosine kinase as philadelphia
chromosome-positive acute lymphoblastic leukemia treatment. Expert
Opinion on Drug Discovery \textbf{11}:1061--1070.
doi:\href{https://doi.org/10.1080/17460441.2016.1227318}{10.1080/17460441.2016.1227318}.

\hypertarget{ref-normalization_Sze2017}{}
27. \textbf{Sze MA}, \textbf{Baxter NT}, \textbf{Ruffin MT},
\textbf{Rogers MAM}, \textbf{Schloss PD}. 2017. Normalization of the
microbiota in patients after treatment for colonic lesions. Microbiome
\textbf{5}.
doi:\href{https://doi.org/10.1186/s40168-017-0366-3}{10.1186/s40168-017-0366-3}.

\hypertarget{ref-Flynn2016}{}
28. \textbf{Flynn KJ}, \textbf{Baxter NT}, \textbf{Schloss PD}. 2016.
Metabolic and community synergy of oral bacteria in colorectal cancer.
mSphere \textbf{1}.
doi:\href{https://doi.org/10.1128/msphere.00102-16}{10.1128/msphere.00102-16}.

\hypertarget{ref-normalization_sze2017}{}
29. \textbf{Sze MA}, \textbf{Baxter NT}, \textbf{Ruffin MT},
\textbf{Rogers MAM}, \textbf{Schloss PD}. 2017. Normalization of the
microbiota in patients after treatment for colonic lesions. Microbiome
\textbf{5}.
doi:\href{https://doi.org/10.1186/s40168-017-0366-3}{10.1186/s40168-017-0366-3}.

\hypertarget{ref-crc_model_baxter2016}{}
30. \textbf{Baxter NT}, \textbf{Ruffin MT}, \textbf{Rogers MAM},
\textbf{Schloss PD}. 2016. Microbiota-based model improves the
sensitivity of fecal immunochemical test for detecting colonic lesions.
Genome Medicine \textbf{8}.
doi:\href{https://doi.org/10.1186/s13073-016-0290-3}{10.1186/s13073-016-0290-3}.

\hypertarget{ref-Schloss2009}{}
31. \textbf{Schloss PD}, \textbf{Westcott SL}, \textbf{Ryabin T},
\textbf{Hall JR}, \textbf{Hartmann M}, \textbf{Hollister EB},
\textbf{Lesniewski RA}, \textbf{Oakley BB}, \textbf{Parks DH},
\textbf{Robinson CJ}, \textbf{Sahl JW}, \textbf{Stres B},
\textbf{Thallinger GG}, \textbf{Horn DJV}, \textbf{Weber CF}. 2009.
Introducing mothur: Open-source, platform-independent,
community-supported software for describing and comparing microbial
communities. Applied and Environmental Microbiology
\textbf{75}:7537--7541.
doi:\href{https://doi.org/10.1128/aem.01541-09}{10.1128/aem.01541-09}.

\hypertarget{ref-Kozich2013}{}
32. \textbf{Kozich JJ}, \textbf{Westcott SL}, \textbf{Baxter NT},
\textbf{Highlander SK}, \textbf{Schloss PD}. 2013. Development of a
dual-index sequencing strategy and curation pipeline for analyzing
amplicon sequence data on the MiSeq illumina sequencing platform.
Applied and Environmental Microbiology \textbf{79}:5112--5120.
doi:\href{https://doi.org/10.1128/aem.01043-13}{10.1128/aem.01043-13}.

\hypertarget{ref-Quast2012}{}
33. \textbf{Quast C}, \textbf{Pruesse E}, \textbf{Yilmaz P},
\textbf{Gerken J}, \textbf{Schweer T}, \textbf{Yarza P}, \textbf{Peplies
J}, \textbf{Glöckner FO}. 2012. The SILVA ribosomal RNA gene database
project: Improved data processing and web-based tools. Nucleic Acids
Research \textbf{41}:D590--D596.
doi:\href{https://doi.org/10.1093/nar/gks1219}{10.1093/nar/gks1219}.

\hypertarget{ref-Edgar2011}{}
34. \textbf{Edgar RC}, \textbf{Haas BJ}, \textbf{Clemente JC},
\textbf{Quince C}, \textbf{Knight R}. 2011. UCHIME improves sensitivity
and speed of chimera detection. Bioinformatics \textbf{27}:2194--2200.
doi:\href{https://doi.org/10.1093/bioinformatics/btr381}{10.1093/bioinformatics/btr381}.

\hypertarget{ref-opticlust_westcott2017}{}
35. \textbf{Westcott SL}, \textbf{Schloss PD}. 2017. OptiClust, an
improved method for assigning amplicon-based sequence data to
operational taxonomic units. mSphere \textbf{2}:e00073--17.
doi:\href{https://doi.org/10.1128/mspheredirect.00073-17}{10.1128/mspheredirect.00073-17}.

\hypertarget{ref-Hannigan2017}{}
36. \textbf{Hannigan GD}, \textbf{Duhaime MB}, \textbf{Ruffin MT},
\textbf{Koumpouras CC}, \textbf{Schloss PD}. 2017. Diagnostic potential
\& the interactive dynamics of the colorectal cancer virome.
doi:\href{https://doi.org/10.1101/152868}{10.1101/152868}.

\hypertarget{ref-Li2015}{}
37. \textbf{Li D}, \textbf{Liu C-M}, \textbf{Luo R}, \textbf{Sadakane
K}, \textbf{Lam T-W}. 2015. MEGAHIT: An ultra-fast single-node solution
for large and complex metagenomics assembly via succinct de bruijn
graph. Bioinformatics \textbf{31}:1674--1676.
doi:\href{https://doi.org/10.1093/bioinformatics/btv033}{10.1093/bioinformatics/btv033}.

\hypertarget{ref-Hyatt2010}{}
38. \textbf{Hyatt D}, \textbf{Chen G-L}, \textbf{LoCascio PF},
\textbf{Land ML}, \textbf{Larimer FW}, \textbf{Hauser LJ}. 2010.
Prodigal: Prokaryotic gene recognition and translation initiation site
identification. BMC Bioinformatics \textbf{11}:119.
doi:\href{https://doi.org/10.1186/1471-2105-11-119}{10.1186/1471-2105-11-119}.

\hypertarget{ref-Buchfink2014}{}
39. \textbf{Buchfink B}, \textbf{Xie C}, \textbf{Huson DH}. 2014. Fast
and sensitive protein alignment using DIAMOND. Nature Methods
\textbf{12}:59--60.
doi:\href{https://doi.org/10.1038/nmeth.3176}{10.1038/nmeth.3176}.

\hypertarget{ref-Steinegger2017}{}
40. \textbf{Steinegger M}, \textbf{Söding J}. 2017. MMseqs2 enables
sensitive protein sequence searching for the analysis of massive data
sets. Nature Biotechnology.
doi:\href{https://doi.org/10.1038/nbt.3988}{10.1038/nbt.3988}.

\hypertarget{ref-Kanehisa2015}{}
41. \textbf{Kanehisa M}, \textbf{Sato Y}, \textbf{Kawashima M},
\textbf{Furumichi M}, \textbf{Tanabe M}. 2015. KEGG as a reference
resource for gene and protein annotation. Nucleic Acids Research
\textbf{44}:D457--D462.
doi:\href{https://doi.org/10.1093/nar/gkv1070}{10.1093/nar/gkv1070}.

\hypertarget{ref-scfa_baxter2014}{}
42. \textbf{Baxter NT}, \textbf{Zackular JP}, \textbf{Chen GY},
\textbf{Schloss PD}. 2014. Structure of the gut microbiome following
colonization with human feces determines colonic tumor burden.
Microbiome \textbf{2}:20.
doi:\href{https://doi.org/10.1186/2049-2618-2-20}{10.1186/2049-2618-2-20}.

\hypertarget{ref-caret_citation}{}
43. \textbf{Jed Wing MKC from}, \textbf{Weston S}, \textbf{Williams A},
\textbf{Keefer C}, \textbf{Engelhardt A}, \textbf{Cooper T},
\textbf{Mayer Z}, \textbf{Kenkel B}, \textbf{R Core Team},
\textbf{Benesty M}, \textbf{Lescarbeau R}, \textbf{Ziem A},
\textbf{Scrucca L}, \textbf{Tang Y}, \textbf{Candan C}, \textbf{Hunt.
T}. 2017. Caret: Classification and regression training.

\hypertarget{ref-r_citation_2017}{}
44. \textbf{R Core Team}. 2017. R: A language and environment for
statistical computing. R Foundation for Statistical Computing, Vienna,
Austria.

\hypertarget{ref-benjamini_controlling_1995}{}
45. \textbf{Benjamini Y}, \textbf{Hochberg Y}. 1995. Controlling the
false discovery rate: A practical and powerful approach to multiple
testing. Journal of the Royal Statistical Society Series B
(Methodological) \textbf{57}:289--300.

\newpage

\textbf{Figure 1. No change in SCFA measurements was observed between
normal, adenoma, and carcinoma individuals using HPLC.} Acetate
concentrations in fecal samples of individuals without colon tumors,
adenomas, and carcinomas (A). Butyrate concentrations in fecal samples
of individuals without colon tumors, adenomas, and carcinomas (B).
Propionate concentrations in fecal samples of individuals without colon
tumors, adenomas, and carcinomas (C). The black lines indicate the
median SCFA concentration. Acetate concentrations in fecal samples
before and after treatment for adenoma (yellow) and carcinoma (red) (D).
Butyrate concentrations in fecal samples before and after treatment for
adenoma (yellow) and carcinoma (red) (E). Propionate concentrations in
fecal samples before and after treatment for adenoma (yellow) and
carcinoma (red) (F). The black dots and lines represent the median
change in SCFA concentration.

\textbf{Figure 2. No change in butyrate producing genes identified
between normal, adenoma, and carcinoma individuals.} Imputed gene
relative abundance of important butyrate pathway genes using PICRUSt
(A). Counts per million (corrected for size and number of contigs in an
OPF) for the Butyrate Kinase gene (B). The other butyrate pathway genes
from the PICRUSt analysis did not align to any of the OPFs in the
metagenome analysis.

\textbf{Figure 3. Patients with adenomas had the higest number of
significant negative correlations between OTU relative abundance and
SCFA concentration.} Colors denote the family or lowest taxonomic ID
that an OTU classified to. Fewer significant positive correlations were
observed overall. Additionally, the differences in the number of
significant positive correlations between patients with adenomas versus
individuals without tumors (normal) and patients with carcinomas was not
as pronounced as the number of significant negative correlations.

\textbf{Figure 4. SCFA concentrations do not improve OTU-based Random
Forest models.} The area under the curve of 100 different 80/20 split
OTU-based normal versus adenoma 10-fold CV models with and without SCFAs
(A). The top 10 most important OTUs or SCFAs in the SCFA and OTU adenoma
model (B). The top 10 most important OTUs in the OTU adenoma model (C).
The area under the curve of 100 different 80/20 OTU-based normal versus
carcinoma 10-fold CV models with and without SCFAs (D). The top 10 most
important OTUs or SCFAs in the SCFA and OTU carcinoma model (E). The top
10 most important OTUs in the OTU carcinoma model (F). For (A) and (D)
the black line represents the median AUC. The dotted line highlights an
AUC of 0.5.

\textbf{Figure 5. OTU-based regression Random Forest models of SCFA
concentrations.} The train and test correlation between actual and
predicted values from 100 different 80/20 split OTU-based models with
10-fold CV using regression Random Forest (A). The model accuracy of
predicted SCFA concentrations differed between individuals without
tumors, patients with adenomas, and patients with carcinomas. Generally,
patients with carcinomas had predicted concentrations closest to their
actual measured concentration (B). The top 10 OTUs based on mean
decrease in accuracy (MDA) for each SCFA model, colored by their lowest
taxonomic identification (C).

\newpage

\textbf{Figure S1. Patients with adenomas had the higest number of
significant differences in OTU relative abundance between high/low SCFA
groups.} Colors denote the family or lowest taxonomic ID that an OTU
classified to. Fewer significant OTUs were observed in individuals
without tumors (normal) and patients with carcinomas versus patients
with adenomas.

\textbf{Figure S2. OTU-based classification Random Forest models of
high/low SCFA groups based on overall SCFA median concentration.} The
train and test results of 100 different 80/20 split OTU-based models
with 10-fold CV based on higher or lower than the median SCFA
concentration using classification Random Forest (A). The model accuracy
of predicted high/low SCFA groups differed between individuals without
tumors, patients with adenomas, and patients with carcinomas. Patients
with adenomas consistently had the best classification accuracy (B). The
top 10 OTUs based on mean decrease in accuracy (MDA) for each SCFA
model, colored by their lowest taxonomic identification (C).


\end{document}
