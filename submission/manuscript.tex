\documentclass[11pt,]{article}
\usepackage{lmodern}
\usepackage{amssymb,amsmath}
\usepackage{ifxetex,ifluatex}
\usepackage{fixltx2e} % provides \textsubscript
\ifnum 0\ifxetex 1\fi\ifluatex 1\fi=0 % if pdftex
  \usepackage[T1]{fontenc}
  \usepackage[utf8]{inputenc}
\else % if luatex or xelatex
  \ifxetex
    \usepackage{mathspec}
  \else
    \usepackage{fontspec}
  \fi
  \defaultfontfeatures{Ligatures=TeX,Scale=MatchLowercase}
\fi
% use upquote if available, for straight quotes in verbatim environments
\IfFileExists{upquote.sty}{\usepackage{upquote}}{}
% use microtype if available
\IfFileExists{microtype.sty}{%
\usepackage{microtype}
\UseMicrotypeSet[protrusion]{basicmath} % disable protrusion for tt fonts
}{}
\usepackage[margin=1.0in]{geometry}
\usepackage{hyperref}
\hypersetup{unicode=true,
            pdftitle={Revisiting Short-Chain Fatty Acids and the Microbiota in Colorectal Cancer},
            pdfborder={0 0 0},
            breaklinks=true}
\urlstyle{same}  % don't use monospace font for urls
\usepackage{graphicx,grffile}
\makeatletter
\def\maxwidth{\ifdim\Gin@nat@width>\linewidth\linewidth\else\Gin@nat@width\fi}
\def\maxheight{\ifdim\Gin@nat@height>\textheight\textheight\else\Gin@nat@height\fi}
\makeatother
% Scale images if necessary, so that they will not overflow the page
% margins by default, and it is still possible to overwrite the defaults
% using explicit options in \includegraphics[width, height, ...]{}
\setkeys{Gin}{width=\maxwidth,height=\maxheight,keepaspectratio}
\IfFileExists{parskip.sty}{%
\usepackage{parskip}
}{% else
\setlength{\parindent}{0pt}
\setlength{\parskip}{6pt plus 2pt minus 1pt}
}
\setlength{\emergencystretch}{3em}  % prevent overfull lines
\providecommand{\tightlist}{%
  \setlength{\itemsep}{0pt}\setlength{\parskip}{0pt}}
\setcounter{secnumdepth}{0}
% Redefines (sub)paragraphs to behave more like sections
\ifx\paragraph\undefined\else
\let\oldparagraph\paragraph
\renewcommand{\paragraph}[1]{\oldparagraph{#1}\mbox{}}
\fi
\ifx\subparagraph\undefined\else
\let\oldsubparagraph\subparagraph
\renewcommand{\subparagraph}[1]{\oldsubparagraph{#1}\mbox{}}
\fi

%%% Use protect on footnotes to avoid problems with footnotes in titles
\let\rmarkdownfootnote\footnote%
\def\footnote{\protect\rmarkdownfootnote}

%%% Change title format to be more compact
\usepackage{titling}

% Create subtitle command for use in maketitle
\newcommand{\subtitle}[1]{
  \posttitle{
    \begin{center}\large#1\end{center}
    }
}

\setlength{\droptitle}{-2em}
  \title{Revisiting Short-Chain Fatty Acids and the Microbiota in Colorectal
Cancer}
  \pretitle{\vspace{\droptitle}\centering\huge}
  \posttitle{\par}
  \author{}
  \preauthor{}\postauthor{}
  \date{}
  \predate{}\postdate{}

\usepackage{helvet} % Helvetica font
\renewcommand*\familydefault{\sfdefault} % Use the sans serif version of the font
\usepackage[T1]{fontenc}

\usepackage[none]{hyphenat}

\usepackage{setspace}
\doublespacing
\setlength{\parskip}{1em}

\usepackage{lineno}

\usepackage{pdfpages}

\begin{document}
\maketitle

\vspace{35mm}

Running title: SCFAs and Colorectal Cancer

\vspace{35mm}

Marc A Sze\({^1}\), Nicholas A Lesniak\({^1}\), Mack T Ruffin
IV\({^2}\), Patrick D. Schloss\({^1}\)\({^\dagger}\)

\vspace{40mm}

\(\dagger\) To whom correspondence should be addressed:
\href{mailto:pschloss@umich.edu}{\nolinkurl{pschloss@umich.edu}}

\(1\) Department of Microbiology and Immunology, University of Michigan,
Ann Arbor, MI 48109

\(2\) Department of Family Medicine and Community Medicine, Penn State
Hershey Medical Center, Hershey, PA

\newpage

\linenumbers

\subsection{Abstract}\label{abstract}

\newpage

\subsection{Introduction}\label{introduction}

\newpage

\subsection{Results}\label{results}

\textbf{Decreased SCFA concentrations are not associated with tumors.}
We used frozen fecal samples from 490 individuals and HPLC to measure
acetate, butyrate, and propionate concentrations at a cross-sectional
point in time. There was no difference between individuals with normal
colons (n=172) and patients with either an adenoma (n=198) or carcinoma
(n=120) for any of the SCFAs measured after multiple comparison
correction (P-value \textgreater{} 0.15) {[}Figure 1A - 1C{]}. We next
measured the concentration of SCFAs in 67 patients with an adenoma
(n=41) or carcinoma (n=26) in which we had pre- and post-treatment fecal
samples. Although there was a general trend for increasing acetate,
butyrate, and propionate concentrations after treatment for tumors,
there was no significant difference pre- and post-treatment for patients
with adenoma (P-value \textgreater{} 0.20) or carcinoma (P-value
\textgreater{} 0.80) {[}Figure 1D - 1F{]}. Even though there was no
change in SCFA concentrations between individuals with normal colons and
those with tumors, this information could still be important to help
classify disease.

\textbf{Random Forest models with SCFA concentrations do not classify
tumors better.} Using the Random Forest algorithm we built models to
classify normal versus adenoma and normal versus carcinoma with OTUs or
OTUs and SCFA concentrations. For both adenoma and carcinoma models,
there was no difference between the median AUC of models with or without
SCFA concentrations (P-value \textgreater{} 0.05) {[}Figure 2{]}.
Although including SCFA concentrations did not add extra information for
classification of disease using Random Forest models, it is possible
that the genes for enzymes involved in SCFA synthesis may change due to
disease status; where a smaller number of microbes are responsible for
the observed SCFA concentrations.

\textbf{Changes in genes for enzymes involved in SCFA synthesis are not
associated with tumors.} Using a list of specific genes that are
important for the synthesis of SCFAs {[}Table S1{]}, we looked for
differences in gene abundance based on individuals having normal colons,
adenomas, or carcinomas. First, using imputed gene relative abundance
based on 16S rRNA gene sequencing we found no difference in any of the
genes inolved with acetate, butyrate, or propionate synthesis (P-value
\textgreater{} 0.90) {[}Table S2{]}. This similarity between groups is
highlighted by visualizing genes important in butyrate synthesis
{[}Figure 3A{]}. Next, we took a subset of these 490 fecal samples
(n=85) and performed metagenomic sequencing to confirm these results.
Like the imputed gene results, metagenomic analysis found that there was
no difference in any of the genes involved in SCFA synthesis between
individuals with normal colons (n=29) and patients with adenoma (n=28)
or carcinoma (n=28) (P-value \textgreater{} 0.70) {[}Table S3{]}. This
lack of difference is highlighted when we visualize the results for
butyrate kinase {[}Figure 3B{]}. These observations provide evidence
that the gene content does not change due to tumors. Although these
results suggest that SCFA concentrations do not change due to tumor,
there may be some errors present within our results that may account for
the lack of associations.

\textbf{Expected taxa are associated with higher SCFA concentrations
regardless of tumor status.} Using OTU data we built Random Forest
models to classify higher than median and lower than median SCFA
concentrations. Overall, OTU data had a reasonable ability to classify
high and low SCFA concentrations {[}Figure S1A{]}. However, these models
tended to be overfit, suggesting that rarer taxa may be important for
this classification {[}Figure S1A{]}. The most important OTUs to these
models (assessed with mean decrease in accuracy (MDA)) were to taxa that
are normally associated with SCFA production {[}Figure S1B{]}. These
results highlight that there are no immediately apparent errors in our
data because SCFA concentrations are associated with taxa known to
produce acetate, butyrate, and propionate. Additionally, OTUs associated
with these taxa are the most important to models that can classify high
and low SCFA concentrations. Overall, our results do not support the
hypothesis that lower SCFA concentrations are associated with colorectal
cancer.

\newpage

\subsection{Discussion}\label{discussion}

\newpage

\subsection{Conclusions}\label{conclusions}

\newpage

\subsection{Materials and Methods}\label{materials-and-methods}

\textbf{Study design and sampling.} The overall protocol has been
described in detail previously (1, 2). In brief, this study used fecal
samples obtained at either a single cross-sectional time point (n=490)
or from before (pre-) and after (post-) treatment for their tumor
(n=67). For patients undergoing treatment for their tumor the length of
time between their initial and follow up sample ranged from 188 - 546
days. Our use of treatment has been previously defined as encompassing
removal of a tumor with or without chemotherapy and radiation (1).
Diagnosis of tumor was made by colonoscopic examination and
histopathological review of biopsies obtained (1, 2). The University of
Michigan Institutional Review Board approved the study and informed
consent was obtained from all participants in accordance to the
guidelines set out by the Helsinki Decleration.

\textbf{Measuring specific SCFAs.} Our protocol for the measurement of
acetate, butyrate, and propionate followed a previously published
protocol (3). The following changes to this protocol included the use of
frozen fecal samples suspended in 1ml of PBS instead of fecal
suspensions in DNA Genotek OmniGut tubes, and the use of the acutal
weight of fecal samples instead of the average weight for SCFA
concentration normalizations. These changes did not affect the overall
median concentrations of these SCFAs between the two studies (see Table
1 (3) and Figure 1 in this report).

\textbf{16s rRNA gene sequencing.} The workflow and processing have been
described previously (1, 4, 5). The major differences from these
previous reports include: the use of version 1.39.5 of the mothur
software package and clustering Operational Taxonomic Units (OTUs) at
97\% similarity used the OptClust algorithm (6).

\textbf{Generating imputed metagenomes.} The use of PICRUSt version
1.1.2 with the recommended standard operating protocol (7) was used.
Briefly, the mothur shared file and metadata was converted into a biom
formated table using the biom convert function, the subsequent biom file
was processed with the normalize\_by\_copy\_number.py function, and
subsequent imputed metagenomes created using the predict\_metagenomes.py
function.

\textbf{Obtaining OPFs from metagenomes.} A subset of the
cross-sectional group (n=490) containing a total of 85 individuals
(normal n=29 normal, adenoma n=28, and carcinoma n=28) was shotgun
sequenced on an Illumina HiSeq using 125 bp paired end reads and a
previously described method (8). Briefly, the sequences were quality
filtered and sequences aligning to the human genome were removed prior
to contig assembly with MEGAHIT (9). Open Reading Frames (ORFs) were
identified using Prodigal (10), counts generated using Diamond (11),
subsequent clustering into Operational Protein Families (OPFs) used
mmseq2 (12), and OPF alignment used the KEGG database (13).

\textbf{Pulling genes involved with SCFA synthesis.} Specific genes
located near the end of the pathways involved in the synthesis of
acetate, butyrate, and propionate were analyzed for any differences
between individuals with normal colons and those with tumors. These
genes were based on pathways from KEGG as well as previous research (13,
14) and a list can be found in the supplemental material {[}Table S1{]}.

\textbf{Random Forest Models.} The model was first trained on 80\% of
the data and then tested on the held out 20\% (80/20 split) using the
Random Forest algorithm for both classification and regression models
(15). This was repeated on 100 differen 80/20 splits of the data to
generate a reasonable range for the AUC of the model. The reported AUCs,
unless otherwise specified, are for the test sets. Classification models
were built to group normal versus adenoma and normal versus carcinoma or
high versus low SCFA concentrations.

\textbf{Statistical analysis workflow.} All analysis was performed using
the statistical language R (16). Generally, differences between the
different disease groups used a Kruskal-Walis rank sum test with a
Dunn's post-hoc test. Where appropriate Benjamini-Hochberg was used to
correct for multiple comparisons (17). First, we assessed differences in
SCFA concentrations measured by HPLC between individuals with normal
colons and patients with tumors (adenoma or carcinoma). We then analyzed
whether SCFA concentrations changed in patients with an adenoma or
carcinoma pre- versus post-treatment. Next, we assessed whether OTUs
alone or OTUs and SCFAs were better able to classify individuals with
and without tumor using Random Forest models. Next, the imputed gene
counts of important mediators of SCFA creation were tested. Finally, the
counts generated for OPFs that matched important genes involved with
SCFA creation were analyzed. Finally, models to classify high or low
SCFA concentration based on 16S rRNA gene sequencing data were created
using the Random Forest algorithm.

\newpage

\subsection{Acknowledgements}\label{acknowledgements}

The authors thank the Great Lakes-New England Early Detection Research
Network for providing the fecal samples that were used in this study. We
would also like to thank Kwi Kim and Thomas M Schmidt for their help in
running the short-chian fatty acid analysis on the High-Performace
Liquid Chromatography machine at the University of Michigan. Salary
support for Marc A. Sze came from the Canadian Institute of Health
Research and NIH grant UL1TR002240. Salary support for Patrick D.
Schloss came from NIH grants P30DK034933 and 1R01CA215574.

\newpage

\subsection{References}\label{references}

\hypertarget{refs}{}
\hypertarget{ref-normalization_sze2017}{}
1. \textbf{Sze MA}, \textbf{Baxter NT}, \textbf{Ruffin MT},
\textbf{Rogers MAM}, \textbf{Schloss PD}. 2017. Normalization of the
microbiota in patients after treatment for colonic lesions. Microbiome
\textbf{5}.
doi:\href{https://doi.org/10.1186/s40168-017-0366-3}{10.1186/s40168-017-0366-3}.

\hypertarget{ref-crc_model_baxter2016}{}
2. \textbf{Baxter NT}, \textbf{Ruffin MT}, \textbf{Rogers MAM},
\textbf{Schloss PD}. 2016. Microbiota-based model improves the
sensitivity of fecal immunochemical test for detecting colonic lesions.
Genome Medicine \textbf{8}.
doi:\href{https://doi.org/10.1186/s13073-016-0290-3}{10.1186/s13073-016-0290-3}.

\hypertarget{ref-scfa_measures_venkataraman2016}{}
3. \textbf{Venkataraman A}, \textbf{Sieber JR}, \textbf{Schmidt AW},
\textbf{Waldron C}, \textbf{Theis KR}, \textbf{Schmidt TM}. 2016.
Variable responses of human microbiomes to dietary supplementation with
resistant starch. Microbiome \textbf{4}.
doi:\href{https://doi.org/10.1186/s40168-016-0178-x}{10.1186/s40168-016-0178-x}.

\hypertarget{ref-Schloss2009}{}
4. \textbf{Schloss PD}, \textbf{Westcott SL}, \textbf{Ryabin T},
\textbf{Hall JR}, \textbf{Hartmann M}, \textbf{Hollister EB},
\textbf{Lesniewski RA}, \textbf{Oakley BB}, \textbf{Parks DH},
\textbf{Robinson CJ}, \textbf{Sahl JW}, \textbf{Stres B},
\textbf{Thallinger GG}, \textbf{Horn DJV}, \textbf{Weber CF}. 2009.
Introducing mothur: Open-source, platform-independent,
community-supported software for describing and comparing microbial
communities. Applied and Environmental Microbiology
\textbf{75}:7537--7541.
doi:\href{https://doi.org/10.1128/aem.01541-09}{10.1128/aem.01541-09}.

\hypertarget{ref-Kozich2013}{}
5. \textbf{Kozich JJ}, \textbf{Westcott SL}, \textbf{Baxter NT},
\textbf{Highlander SK}, \textbf{Schloss PD}. 2013. Development of a
dual-index sequencing strategy and curation pipeline for analyzing
amplicon sequence data on the MiSeq illumina sequencing platform.
Applied and Environmental Microbiology \textbf{79}:5112--5120.
doi:\href{https://doi.org/10.1128/aem.01043-13}{10.1128/aem.01043-13}.

\hypertarget{ref-opticlust_westcott2017}{}
6. \textbf{Westcott SL}, \textbf{Schloss PD}. 2017. OptiClust, an
improved method for assigning amplicon-based sequence data to
operational taxonomic units. mSphere \textbf{2}:e00073--17.
doi:\href{https://doi.org/10.1128/mspheredirect.00073-17}{10.1128/mspheredirect.00073-17}.

\hypertarget{ref-Langille2013}{}
7. \textbf{Langille MGI}, \textbf{Zaneveld J}, \textbf{Caporaso JG},
\textbf{McDonald D}, \textbf{Knights D}, \textbf{Reyes JA},
\textbf{Clemente JC}, \textbf{Burkepile DE}, \textbf{Thurber RLV},
\textbf{Knight R}, \textbf{Beiko RG}, \textbf{Huttenhower C}. 2013.
Predictive functional profiling of microbial communities using 16S rRNA
marker gene sequences. Nature Biotechnology \textbf{31}:814--821.
doi:\href{https://doi.org/10.1038/nbt.2676}{10.1038/nbt.2676}.

\hypertarget{ref-Hannigan2017}{}
8. \textbf{Hannigan GD}, \textbf{Duhaime MB}, \textbf{Ruffin MT},
\textbf{Koumpouras CC}, \textbf{Schloss PD}. 2017. Diagnostic potential
\& the interactive dynamics of the colorectal cancer virome.
doi:\href{https://doi.org/10.1101/152868}{10.1101/152868}.

\hypertarget{ref-Li2015}{}
9. \textbf{Li D}, \textbf{Liu C-M}, \textbf{Luo R}, \textbf{Sadakane K},
\textbf{Lam T-W}. 2015. MEGAHIT: An ultra-fast single-node solution for
large and complex metagenomics assembly via succinct de bruijn graph.
Bioinformatics \textbf{31}:1674--1676.
doi:\href{https://doi.org/10.1093/bioinformatics/btv033}{10.1093/bioinformatics/btv033}.

\hypertarget{ref-Hyatt2010}{}
10. \textbf{Hyatt D}, \textbf{Chen G-L}, \textbf{LoCascio PF},
\textbf{Land ML}, \textbf{Larimer FW}, \textbf{Hauser LJ}. 2010.
Prodigal: Prokaryotic gene recognition and translation initiation site
identification. BMC Bioinformatics \textbf{11}:119.
doi:\href{https://doi.org/10.1186/1471-2105-11-119}{10.1186/1471-2105-11-119}.

\hypertarget{ref-Buchfink2014}{}
11. \textbf{Buchfink B}, \textbf{Xie C}, \textbf{Huson DH}. 2014. Fast
and sensitive protein alignment using DIAMOND. Nature Methods
\textbf{12}:59--60.
doi:\href{https://doi.org/10.1038/nmeth.3176}{10.1038/nmeth.3176}.

\hypertarget{ref-Steinegger2017}{}
12. \textbf{Steinegger M}, \textbf{Söding J}. 2017. MMseqs2 enables
sensitive protein sequence searching for the analysis of massive data
sets. Nature Biotechnology.
doi:\href{https://doi.org/10.1038/nbt.3988}{10.1038/nbt.3988}.

\hypertarget{ref-Kanehisa2015}{}
13. \textbf{Kanehisa M}, \textbf{Sato Y}, \textbf{Kawashima M},
\textbf{Furumichi M}, \textbf{Tanabe M}. 2015. KEGG as a reference
resource for gene and protein annotation. Nucleic Acids Research
\textbf{44}:D457--D462.
doi:\href{https://doi.org/10.1093/nar/gkv1070}{10.1093/nar/gkv1070}.

\hypertarget{ref-scfa_baxter2014}{}
14. \textbf{Baxter NT}, \textbf{Zackular JP}, \textbf{Chen GY},
\textbf{Schloss PD}. 2014. Structure of the gut microbiome following
colonization with human feces determines colonic tumor burden.
Microbiome \textbf{2}:20.
doi:\href{https://doi.org/10.1186/2049-2618-2-20}{10.1186/2049-2618-2-20}.

\hypertarget{ref-randomforest_citation_2002}{}
15. \textbf{Liaw A}, \textbf{Wiener M}. 2002. Classification and
regression by randomForest. R News \textbf{2}:18--22.

\hypertarget{ref-r_citation_2017}{}
16. \textbf{R Core Team}. 2017. R: A language and environment for
statistical computing. R Foundation for Statistical Computing, Vienna,
Austria.

\hypertarget{ref-benjamini_controlling_1995}{}
17. \textbf{Benjamini Y}, \textbf{Hochberg Y}. 1995. Controlling the
false discovery rate: A practical and powerful approach to multiple
testing. Journal of the Royal Statistical Society Series B
(Methodological) \textbf{57}:289--300.

\newpage

\textbf{Figure 1. Using HPLC no change in SCFA measurements was observed
between normal, adenoma, and carcinoma individuals.} Acetate
concentrations in fecal samples of individuals with normal colons,
adenomas, and carcinomas (A). Butyrate concentrations in fecal samples
of individuals with normal colons, adenomas, and carcinomas (B).
Propionate concentrations in fecal samples of individuals with normal
colons, adenomas, and carcinomas (C). The black links indicate the
median SCFA concentration. Acetate concentrations in fecal samples
before and after treatment for adenoma (yellow) and carcinoma (red) (D).
Butyrate concentrations in fecal samples before and after treatment for
adenoma (yellow) and carcinoma (red) (E). Propionate concentrations in
fecal samples before and after treatment for adenoma (yellow) and
carcinoma (red) (F). The black dots and lines represent the median
change in SCFA concentration.

\textbf{Figure 2. SCFAs do not improve OTU-based Random Forest models.}
Difference between the area under the curve of 100 different 80/20 split
OTU-based normal versus adenoma 10-fold CV models with and without SCFAs
(A). Difference between the area under the curve of 100 different 80/20
OTU-based normal versus carcinoma 10-fold CV models with and without
SCFAs (B). The black linke represents the median AUC. The dotted line
highlights an AUC of 0.5.

\textbf{Figure 3. No change in butyrate producing genes identified
between normal, adenoma, and carcinoma individuals.} Imputed gene
relative abundance of important butyrate pathway genes using PICRUSt
(A). Counts per million (corrected for size and number of contigs in an
OPF) for the Butyrate Kinase gene (B). The other genes from the PICRUSt
analysis did not align to any of the OPFs in the metagenome analysis.

\newpage

\textbf{Figure S1. OTU-based Random Forest models of SCFA
concentrations.} Classification Random Forest train and tests of 100
different 80/20 OTU-based models with 10-fold CV based on higher or
lower than the medain SCFA concentration (A). The top 10 OTUs based on
mean decrease in accuracy (MDA) for each model, colored by their lowest
taxonomic identification (B). Regression Random Forest train and tests
of 100 different 80/20 OTU-based models with 10-fold CV based on
correlation to actual SCFA concentration (C). The top 10 OTUs based on
mean decrease in accuracy (MDA) for each model, colored by their lowest
taxonomic identification (D).


\end{document}
